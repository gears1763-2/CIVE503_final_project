\documentclass[conference]{IEEEtran}
%\IEEEoverridecommandlockouts
% The preceding line is only needed to identify funding in the first footnote. If that is unneeded, please comment it out.
\usepackage{cite}
\usepackage{amsmath,amssymb,amsfonts}
\usepackage{algorithmic}
\usepackage{graphicx}
	\graphicspath{
		{figures}
	}
\usepackage{lineno}
\usepackage{textcomp}
\usepackage{xcolor}
\def\BibTeX{
	{
		\rm B\kern-.05em{\sc i\kern-.025em b}
		\kern-.08em T\kern-.1667em\lower.7ex
		\hbox{E}\kern-.125emX
    }
}


\begin{document}

%%%% ==== TITLE & AUTHORSHIP ======================================================== %%%%

\title{Efficient Sampling for Surrogate Modelling}

\author{\IEEEauthorblockN{1\textsuperscript{st} Anthony Truelove}
\IEEEauthorblockA{\textit{Pacific Regional Institute for Marine Energy Discovery (PRIMED)} \\
\textit{Institute for Integrated Energy Systems (IESVic) - University of Victoria}\\
Victoria, Canada \\
wtruelove@uvic.ca}}

\maketitle

\thispagestyle{plain}
\pagestyle{plain}


%%%% ==== ABSTRACT & KEYWORDS ======================================================= %%%%

\begin{abstract}
[...]
\end{abstract}

\begin{IEEEkeywords}
sampling, optimization, surrogate modelling
\end{IEEEkeywords}


%%%% ==== MAIN ====================================================================== %%%%

%\linenumbers
\section{Introduction}

	Consider an optimization problem in which the objective function is computationally expensive to evaluate, so much so that applying an optimization algorithm to the objective directly is intractable. For example, in model-based design optimization, the objective function is often high-resolution modelling and simulation software that can be used to assess the performance of candidate designs. However, when individual model runs take anywhere from minutes to days to complete, these models do not scale well in the context of optimization (where algorithms often evaluate the objective thousands of times [or more!] in search of optima). This then motivates the concept of a \textit{surrogate model}: an approximation of a more expensive model that seeks to minimize computational expense while retaining a maximum of model fidelity \cite{Forrester_2008}.

	Surrogate modelling has taken hold in a variety of domains in recent years. For example, \cite{Westermann_2019} reviews the application of surrogate modelling to sustainable building design. An example is provided in \cite{Liu_2023} of applying surrogate modelling to the optimal design of combustion systems, while an example is provided in \cite{Haghi_2022} of applying surrogate modelling to the optimal design of wind turbines. Yet another example is provided in \cite{vanderHoog_2018} of applying surrogate modelling to the optimization of policy and decision-making in the domain of economics. Compound these examples with the utility of machine learning models as surrogates (as mentioned in all of \cite{Westermann_2019, Liu_2023, vanderHoog_2018}) and it seems that surrogate modelling will remain a useful technique (and hence a relevant topic) for the foreseeable future.
	
	Of course, in order to construct a surrogate model for any given use case, some amount of data is required (which implies sampling the problem space via the computationally expensive objective function). As such, there exists the notion of surrogate utility versus surrogate cost (i.e., a \textit{surrogate efficiency}), and this begs two questions:
	
\begin{enumerate}
	\item Is there a general metric for surrogate efficiency?
	\item Is there a clear ``winner" in terms of sampling scheme?
\end{enumerate}

\noindent This paper aims to address both questions.

\section{Methodology}

\subsection{General Surrogate Efficiency Metric: Concepts}

The concept of ``sampling efficiency" in the context of surrogate modelling is mentioned (but never precisely defined) in all of \cite{Gong_2017, Westermann_2019_2, Yin_2011}. Similarly searching the literature for references to ``surrogate efficiency" yields a single relevant result \cite{Casper_2016} (but again, it is not precisely defined). Therefore, a novel and general surrogate efficiency metric $\eta_\textrm{SM}$ is proposed as part of this work.

\subsubsection{Logic}

Any measure of surrogate efficiency should express the trade-off between surrogate utility (i.e., how accurate/precise is the surrogate?) and surrogate cost (i.e., what is the computational expense to build the surrogate in the first place?). This logic is sketched out in (\ref{eqn:efficiency_concept}).

\begin{equation}
	\eta_\textrm{SM} \sim \frac{\textrm{surrogate utility}}{\textrm{surrogate cost}}
	\label{eqn:efficiency_concept}
\end{equation}

\subsubsection{Surrogate Utility}

If surrogate utility is essentially a measure of surrogate accuracy and precision, then one might say that utility is inversely proportional to error (\ref{eqn:surrogate_utility_concept}).

\begin{equation}
	\textrm{surrogate utility} \propto \frac{1}{\textrm{surrogate error}}
	\label{eqn:surrogate_utility_concept}
\end{equation}

\noindent For the sake of generality, a normalized error metric is desirable. To that end, the damped absolute percentage error (d-APE) metric is selected in this work. As per \cite{Rulff_2024}, d-APE can be expressed as

\begin{equation}
	\textrm{d-APE} = \begin{cases}
		\left|\frac{\widehat{y} - y}{T}\right| & \textrm{if}\;\;|y|\leq T \\
		{} & {} \\
		\left|\frac{\widehat{y} - y}{y}\right| & \textrm{otherwise}
	\end{cases}
	\label{eqn:d-APE}
\end{equation}

\noindent where $\widehat{y}$ is a value predicted by the surrogate, $y$ is the corresponding ``true value", and $T \neq 0$ is some domain-specific threshold. Finally, for any use case (i.e., any set of $\left\{(\widehat{y}_i,\;y_i)\right\}$), surrogate error can be expressed as

\begin{equation}
	\textrm{surrogate error} = \mu_\textrm{d-APE} + \text{IQR}_\textrm{d-APE}
	\label{eqn:surrogate_error}
\end{equation}

\noindent where $\mu_\textrm{d-APE} \geq 0$ is the mean of the d-APE values (i.e., a measure of surrogate accuracy) and $\text{IQR}_\textrm{d-APE} \geq 0$ is the inter-quartile range of the d-APE values (i.e., a measure of surrogate precision).

\subsubsection{Surrogate Cost}

If one assumes that the cost of building a surrogate is dominated by the computational expense of sampling the objective function, then it follows that surrogate cost is proportional to the number of samples. This logic is sketched out in (\ref{eqn:surrogate_cost_concept})

\begin{equation}
	\textrm{surrogate cost} \propto N
	\label{eqn:surrogate_cost_concept}
\end{equation}

\noindent where $N > 0$ is the number of samples (i.e., the number of objective function calls).

\subsection{General Surrogate Efficiency Metric: Proposition}

Given the preceding concepts, the following expression for a general surrogate efficiency metric is proposed:

\begin{equation}
	\eta_\textrm{SM} = \exp\left[-\sqrt[D]{N}(\mu_\textrm{d-APE} + \textrm{IQR}_\textrm{d-APE})\right]
	\label{eqn:efficiency_proposition}
\end{equation}

\noindent where $D>0$ is the problem dimensionality (i.e., number of objective function inputs). Observe that the expression proposed in (\ref{eqn:efficiency_proposition}) has the following desirable properties:

\begin{itemize}
	\item $\eta_\textrm{SM} \in [0,1]$ for any values of $D$, $N$, $\mu_\textrm{d-APE}$, and $\textrm{IQR}_\textrm{d-APE}$.
	\item For any $\sqrt[D]{N} > 0$, $\eta_\textrm{SM} \to 1$ as $\mu_\textrm{d-APE} + \textrm{IQR}_\textrm{d-APE} \to 0$. That is, increasing surrogate utility is rewarded.
	\item For any $\mu_\textrm{d-APE} + \textrm{IQR}_\textrm{d-APE} > 0$, $\eta_\textrm{SM} \to 0$ as $\sqrt[D]{N} \to \infty$. That is, increasing surrogate cost is penalized.
\end{itemize}

\subsection{Sampling Schemes}

[...] \textcolor{red}{I think I'll just test a few rather than proposing anything new here. In reviewing the literature, it looks like everyone and their dog is proposing various sampling schemes, but what's lacking is a standardized metric/method for comparing their efficiencies for the purpose of surrogate modelling (as far as I can tell).}

\subsection{Benchmark Problems}

[...]

Pick some $\mathbb{R}^D\;\to\;\mathbb{R}$ problems, so that results for different problem sizes can be obtained \textcolor{red}{I think I'll just do scalar objectives here, as they're a bit easier to interpret (and tend to be more easily scalable in terms of input dimensionality). I'll punt multi-objective stuff to future work.}

\section{Results}

[...]

\section{Discussion}

[...]

\section{Conclusion}

[...]

\section{Future Work}

[...]


%%%% ==== REFERENCES ================================================================ %%%%

\bibliography{/home/primed-anthony/MECH_PhD/tex/refs/refs.bib}{}
\bibliographystyle{IEEEtran}

\end{document}


%%%% ==== TEMPLATES ================================================================= %%%%

\begin{table}[htbp]
	\caption{Table Type Styles}
	\begin{center}
	\begin{tabular}{|c|c|c|c|}
		\hline
		\textbf{Table}&\multicolumn{3}{|c|}{\textbf{Table Column Head}} \\
		\cline{2-4} 
		\textbf{Head} & \textbf{\textit{Table column subhead}}& \textbf{\textit{Subhead}}& 		\textbf{\textit{Subhead}} \\
		\hline
		copy& More table copy$^{\mathrm{a}}$& &  \\
		\hline
		\multicolumn{4}{l}{$^{\mathrm{a}}$Sample of a Table footnote.}
	\end{tabular}
	\label{tab1}
	\end{center}
\end{table}

\begin{figure}[htbp]
	\centerline{\includegraphics{fig1.png}}
	\caption{Example of a figure caption.}
	\label{fig}
\end{figure}
